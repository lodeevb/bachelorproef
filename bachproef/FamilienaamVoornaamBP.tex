%===============================================================================
% LaTeX sjabloon voor de bachelorproef toegepaste informatica aan HOGENT
% Meer info op https://github.com/HoGentTIN/latex-hogent-report
%===============================================================================

\documentclass[dutch,dit,thesis]{hogentreport}

% TODO:
% - If necessary, replace the option `dit`' with your own department!
%   Valid entries are dbo, dbt, dgz, dit, dlo, dog, dsa, soa
% - If you write your thesis in English (remark: only possible after getting
%   explicit approval!), remove the option "dutch," or replace with "english".

\usepackage{lipsum} % For blind text, can be removed after adding actual content

%% Pictures to include in the text can be put in the graphics/ folder
\graphicspath{{graphics/}}

%% For source code highlighting, requires pygments to be installed
%% Compile with the -shell-escape flag!
\usepackage[section]{minted}
%% If you compile with the make_thesis.{bat,sh} script, use the following
%% import instead:
%% \usepackage[section,outputdir=../output]{minted}
\usemintedstyle{solarized-light}
\definecolor{bg}{RGB}{253,246,227} %% Set the background color of the codeframe

%% Change this line to edit the line numbering style:
\renewcommand{\theFancyVerbLine}{\ttfamily\scriptsize\arabic{FancyVerbLine}}

%% Macro definition to load external java source files with \javacode{filename}:
\newmintedfile[javacode]{java}{
    bgcolor=bg,
    fontfamily=tt,
    linenos=true,
    numberblanklines=true,
    numbersep=5pt,
    gobble=0,
    framesep=2mm,
    funcnamehighlighting=true,
    tabsize=4,
    obeytabs=false,
    breaklines=true,
    mathescape=false
    samepage=false,
    showspaces=false,
    showtabs =false,
    texcl=false,
}

% Other packages not already included can be imported here

%%---------- Document metadata -------------------------------------------------
% TODO: Replace this with your own information
\author{Lode Van Beneden}
\supervisor{Mevr. C. De Leenheer}
\cosupervisor{Mr. A. Hantson}
\title[Optionele ondertitel]%
    {In hoeverre kan kunstmatige intelligentie worden in-gezet om vermoeidheid bij automobilisten te detecte-ren en signaleren, en welke maatregelen zouden kun-nen worden geïmplementeerd om ongevallen te voor-komen?}
\academicyear{\advance\year by -1 \the\year--\advance\year by 1 \the\year}
\examperiod{1}
\degreesought{\IfLanguageName{dutch}{Professionele bachelor in de toegepaste informatica}{Bachelor of applied computer science}}
\partialthesis{false} %% To display 'in partial fulfilment'
%\institution{Internshipcompany BVBA.}

%% Add global exceptions to the hyphenation here
\hyphenation{back-slash}

%% The bibliography (style and settings are  found in hogentthesis.cls)
\addbibresource{bachproef.bib}            %% Bibliography file
\addbibresource{../voorstel/voorstel.bib} %% Bibliography research proposal
\defbibheading{bibempty}{}

%% Prevent empty pages for right-handed chapter starts in twoside mode
\renewcommand{\cleardoublepage}{\clearpage}

\renewcommand{\arraystretch}{1.2}

%% Content starts here.
\begin{document}

%---------- Front matter -------------------------------------------------------

\frontmatter

\hypersetup{pageanchor=false} %% Disable page numbering references
%% Render a Dutch outer title page if the main language is English
\IfLanguageName{english}{%
    %% If necessary, information can be changed here
    \degreesought{Professionele Bachelor toegepaste informatica}%
    \begin{otherlanguage}{dutch}%
        \maketitle%
    \end{otherlanguage}%
}{}

%% Generates title page content
\maketitle
\hypersetup{pageanchor=true}

%%=============================================================================
%% Voorwoord
%%=============================================================================

\chapter*{\IfLanguageName{dutch}{Woord vooraf}{Preface}}%
\label{ch:voorwoord}
\setlength{\parskip}{1em}
%% TODO:
%% Het voorwoord is het enige deel van de bachelorproef waar je vanuit je
%% eigen standpunt (``ik-vorm'') mag schrijven. Je kan hier bv. motiveren
%% waarom jij het onderwerp wil bespreken.
%% Vergeet ook niet te bedanken wie je geholpen/gesteund/... heeft
Geachte lezer,

Met veel genoegen presenteer ik mijn bachelorproef, met de titel "In hoeverre kan kunstmatige intelligentie worden in-gezet om vermoeidheid bij automobilisten te detecteren en signaleren, en welke maatregelen zouden kunnen worden geïmplementeerd om ongevallen te voorkomen?".
Dit onderzoek is het resultaat van mijn toewijding aan het vakgebied van kunstmatige intelligentie.

De keuze voor dit onderwerp werd beslist door mijn interesse in de opkomst van kunstmatige intelligentie en de bezorgdheid over het gevaarlijk rijgedrag van sommige automobilisten. In dit voorwoord
wil ik kort de achtergrond van mijn onderzoek uitleggen en mijn dank uitspreken aan degene die mij hebben geholpen aan het tot stand komen van dit werk.

Mijn oprechte dank gaat uit naar mijn begeleider, Alec Hantson, voor de waardevolle begeleiding en ondersteuning gedurende dit onderzoeksproces. Zijn expertise op het vakgebied van kunstmatige intelligentie heeft mij veel begrip en inzicht gegeven.
Ook wil ik mijn promotor, Chloé De Leenheer, bedanken die mij ondersteunt en begeleidt heeft om dit werk zo correct en professioneel mogelijk te verwoorden.

Dit onderzoek heeft niet enkel mijn academische vaardigheid verbreedt, maar heeft mij ook bewust gemaakt op de impact die kunstmatige intelligentie kan hebben op maatschappelijke kwesties zoals verkeersveiligheid.
Ik hoop dat de resultaten en aanbevelingen in deze bachelorproef bijdragen aan de rol van technologie in het verkeerd.

Ik nodig u uit om de pagina's van dit werk te verkennen en hoop dat dit u een inzicht geeft in de relatie tussen kunstmatige intelligentie, vermoeidheid bij automobilisten en verkeersveiligheid.
Hopelijk draagt dit onderzoek bij aan een veiligere en slimmere mobiliteit in het verkeer in de toekomst.

Met vriendelijke groet,

Lode Van Beneden
\\
Bachelor Toegepaste informatica
\\
Hogeschool Gent
Geachte lezer,

Met veel genoegen presenteer ik mijn bachelorproef, met de titel "In hoeverre kan kunstmatige intelligentie worden in-gezet om vermoeidheid bij automobilisten te detecte-ren en signaleren, en welke maatregelen zouden kun-nen worden geïmplementeerd om ongevallen te voor-komen?".
Dit onderzoek is het resultaat van mijn toewijding aan het vakgebeid van kunstmatige intelligentie.

De keuze voor dit onderwerp werd beslist door mijn interesse in de opkomst van kunstmatige intelligentie en de bezorgdheid over het gevaarlijk rijgedrag van sommige automobilisten. In dit voorwoord
wil ik kort de achtergrond van mijn onderzoek uitleggen en mijn dank uitspreken aan degene die mij hebben geholpen aan het tot stand komen van dit werk.

Mijn oprechte dank gaat uit naar mijn begeleider, Alec Hantson, voor de waardevolle begeleiding en ondersteuning gedurende dit onderzoeksproces. Zijn expertise op het vakgebied van kunstmatige intelligentie heeft mij veel begrip en inzicht gegeven.
Ook wil ik mijn promotor, Chloé De Leenheer, bedanken die mij ondersteunt en begeleidt heeft om dit werk zo correct en professioneel mogelijk te verwoorden.

Dit onderzoek heeft niet enkel mijn academische vaardigheid verbreedt, maar heeft mij ook bewust gemaakt op de impact die kunstamtige intelligentie kan hebben op maatschappelijke kwesties zoals verkeersveiligheid.
Ik hoop dat de resultaten en aanbevelingen in deze bachelorproef brijdragen aan de rol van technologie in het verkeerd.

Ik nodig u uit om de pagina's van dit werk te verkennen en hoop dat dit u een inzicht geeft in de relatie tussen kunstmatige intelligentie, vermoeidheid bij automobilisten en verkeersveiligheid.
Hopelijk draagt dit onderzoek bij aan een veiligere en slimmere mobiliteit in het verkeer in de toekomst.

Met vriendelijke groet,

Lode Van Beneden
Bachelor Toegepaste informatica
Hogeschool Generates
\input{samenvatting}

%---------- Inhoud, lijst figuren, ... -----------------------------------------

\tableofcontents

% In a list of figures, the complete caption will be included. To prevent this,
% ALWAYS add a short description in the caption!
%
%  \caption[short description]{elaborate description}
%
% If you do, only the short description will be used in the list of figures

\listoffigures

% If you included tables and/or source code listings, uncomment the appropriate
% lines.
%\listoftables
%\listoflistings

% Als je een lijst van afkortingen of termen wil toevoegen, dan hoort die
% hier thuis. Gebruik bijvoorbeeld de ``glossaries'' package.
% https://www.overleaf.com/learn/latex/Glossaries

%---------- Kern ---------------------------------------------------------------

\mainmatter{}

% De eerste hoofdstukken van een bachelorproef zijn meestal een inleiding op
% het onderwerp, literatuurstudie en verantwoording methodologie.
% Aarzel niet om een meer beschrijvende titel aan deze hoofdstukken te geven of
% om bijvoorbeeld de inleiding en/of stand van zaken over meerdere hoofdstukken
% te verspreiden!

%%=============================================================================
%% Inleiding
%%=============================================================================

\chapter{\IfLanguageName{dutch}{Inleiding}{Introduction}}%
\label{ch:inleiding}

\section{\IfLanguageName{dutch}{Probleemstelling}{Problem Statement}}%
\label{sec:probleemstelling}

In onze moderne en snel voortbewegende wereld is vermoeidheid een groeiend probleem. Dit heeft een grote impact op de verkeersveiligheid. Het aantal ongevallen die in verband staan met vermoeidheid zijn alarmerend hoog. Om dit probleem op te lossen zullen er innovatieve oplossingen en technologieën nodig zijn die de vermoeidheid vroegtijdig kan opsporen om zo te voorkomen dat automobilisten het verkeer in gevaar brengen. 

\section{\IfLanguageName{dutch}{Onderzoeksvraag}{Research question}}%
\label{sec:onderzoeksvraag}

Dit onderzoek zal dus achterhalen in welke mate er kunstmatige intelligentie geïmplementeerd kan worden zodat de vermoeidheid tijdig wordt opgespoord. Vervolgens zal er verder onderzocht worden hoe dat deze mogelijke oplossingen ingezet kunnen worden in de toekomst om zo het aantal ongevallen die veroorzaakt worden door vermoeidheid te verminderen.

\section{\IfLanguageName{dutch}{Onderzoeksdoelstelling}{Research objective}}%
\label{sec:onderzoeksdoelstelling}

Het belangrijkste doel van dit onderzoek is het creëren en beoordelen van een Proof-of-Concept. Deze PoC zal gebruik maken van kunstmatige intelligentie die de vermoeidheid kan detecteren en signaleren. Het uiteindelijke doel is om mogelijke oplossingen te onderzoeken die eventueel deze technologie kan toepassen om zo verkeersongevallen te verminderen. Het onderzoek zal zich specifiek richten op:
\begin{itemize}
    \item \textbf{Proof-of-Concept:}
    Het ontwerpen en implementeren van een AI-systeem die de vermoeidheid kan detecteren door middel van relevante indicatoren zoals gezichtsuitdrukkingen, oogbewegingen, etc.
    \item \textbf{Evaluatie:}
    Het uitvoeren van tests en experimenten om de effectiviteit van de Proof-of-Concept te evalueren in verschillende
    scenario's.
    \item \textbf{Maatregelen:}
    Het onderzoeken naar mogelijke maatregelen op basis van de verzamelde gegevens en resultaten om zo ongevallen te voorkomen.
    \item \textbf{Ethiek en Privacy:}
    Het beschermen van de privacy van automobilisten door middel van ethische en privacy-gerelateerde overwegingen te onderzoeken.
\end{itemize} 

Het onderzoek dient als een basis voor verdere onderzoeken en implementaties op het gebied van de detectie van vermoeidheid tijdens het rijden. De bevindingen zullen bijdragen aan de toekomstige implementatie van kunstmatige intelligentie in het verkeer.

\section{\IfLanguageName{dutch}{Opzet van deze bachelorproef}{Structure of this bachelor thesis}}%
\label{sec:opzet-bachelorproef}

% Het is gebruikelijk aan het einde van de inleiding een overzicht te
% geven van de opbouw van de rest van de tekst. Deze sectie bevat al een aanzet
% die je kan aanvullen/aanpassen in functie van je eigen tekst.

De rest van deze bachelorproef is als volgt opgebouwd:

In Hoofdstuk~\ref{ch:stand-van-zaken} wordt een overzicht gegeven van de stand van zaken binnen het onderzoeksdomein, op basis van een literatuurstudie.

In Hoofdstuk~\ref{ch:methodologie} wordt de methodologie toegelicht en worden de gebruikte onderzoekstechnieken besproken om een antwoord te kunnen formuleren op de onderzoeksvragen.

In Hoofdstuk~\ref{ch:proof-of-concept} wordt de Proof-of-Concept ontwikkelt die gebruik maakt van kunstmatige intelligentie om vermoeidheid te detecteren.

In Hoofdstuk~\ref{ch:testcases} worden er verschillende testcases uitgevoerd die bepaalde automobilisten gaan controleren op vermoeidheid in het verkeer.

In Hoofdstuk~\ref{ch:conclusie}, tenslotte, wordt de conclusie gegeven en een antwoord geformuleerd op de onderzoeksvragen. Daarbij wordt ook een aanzet gegeven voor toekomstig onderzoek binnen dit domein.
\chapter{\IfLanguageName{dutch}{Stand van zaken}{State of the art}}%
\label{ch:stand-van-zaken}

% Tip: Begin elk hoofdstuk met een paragraaf inleiding die beschrijft hoe
% dit hoofdstuk past binnen het geheel van de bachelorproef. Geef in het
% bijzonder aan wat de link is met het vorige en volgende hoofdstuk.

% Pas na deze inleidende paragraaf komt de eerste sectiehoofding.
Om dit onderzoek tot stand te brengen, is er nood aan bepaalde informatie, waaronder over AI-technieken en vermoeidheid zelf. Nadien wordt dit alles verwerkt om het probleem aan te pakken.

\section{Vermoeidheid}
De verkeersveiligheid heeft een grote last van vermoeidheid en slaperigheid. In de literatuur is de definitie van vermoeidheid en slaperigheid anders en bestaat er eigenlijk geen duidelijk onderscheid tussen beide \autocite{RiguelleGoldenbeld}.

Vermoeidheid verwijst naar een weerstand die veroorzaakt wordt door uitputting. Dit heeft het gevolg dat de taken minder efficiënt worden uitgevoerd. Dit kan zowel fysiek (bijvoorbeeld na het sporten of intensief werken) als mentaal zijn (na een veeleisende intellectuele, mentale of psychologische activiteit). Vermoeidheid zorgt er voor dat de taak minder snel wordt afgemaakt of dat de energie zodanig laag ligt dat er aan de volgende taak niet begonnen wordt.

Slaperigheid verwijst naar de moeite om nog wakker te blijven. Het is gekoppeld aan de biologische slaap-waak proces volgens het circadiaan-ritme. Het circadiaan-ritme is een endogene, regelbare schommeling van ongeveer 24 uur (van Latijnse `circa-diem` - ongeveer een dag) van verschillende biologische systemen in het hele lichaam \autocite{Gianni2018}. Slaperigheid heeft dus geen rechtstreeks verband met de uitvoering van een activiteit. Doordat ons menselijk lichaam beschikt over een 'slaapmodus', dit is voornamelijk tussen middernacht en 6 uur, neemt de alertheid bijgevolg af. Dit komt doordat het lichaam, op een cyclus van 24 uur, meer slaap nodig heeft dan op andere momenten.

Hoewel slaperigheid en vermoeidheid logisch niet synchroon verlopen, worden ze vaak gezamenlijk behandeld in de literatuur vanwege hun overeenkomstige gevolgen. Deze twee aparte gevallen kunnen ook tegelijkertijd voorkomen bij iemand. Om het eenvoudig te houden, wordt 'vermoeidheid' waargenomen als slaperigheid.

\section{Oorzaken van vermoeidheid}
Er zijn vijf algemene factoren, namelijk: de tijd besteed aan een taak of werk, slaaptekort, bioritme, monotonie van een taak en individuele kenmerken, die vermoeidheid veroorzaken \autocite{Brown}.
\subsection{De tijd besteed aan een taak of werk}
Een van de mogelijke oorzaken is de tijd die iemand besteed aan een bepaalde taak of werk. De meeste mensen voelen de eerste symptomen van fysieke vermoeidheid na ongeveer 2 à 3 uren aaneengesloten autorijden \autocite{RiguelleGoldenbeld}.
\subsection{Slaaptekort}
Een slaaptekort kan chronisch of acuut zijn \autocite{VanSchagen2003}. Een chronisch slaaptekort kan het veroorzaakt worden door gevolg van te weinig slaap over een lange periode. Men heeft gemiddeld nood aan 8 uur slaap. Echter is het ook van belang dat er goed geslapen wordt. Er kunnen regelmatige verstoringen optreden tijdens het slapen, wat ook leidt tot een chronisch slaaptekort. Een acuut slaaptekort komt ook door het te weinig slapen, maar is minder structureel. Na één slechte of korte nacht, wordt er al gesproken van een partieel acuut slaaptekort. Een volledig acuut slaaptekort komt pas voor indien er in de gehele 24-uursperiode niet geslapen is.
\subsection{Bioritme}
Het bioritme regelt het slaap-waakritme van de mens. Dit is bij iedereen anders. Het hangt dus samen met de dagelijkse slaapcyclus. Dit betekent dat de mens in de ochtend minder behoefte heeft aan slaapt dan op andere tijdstippen. Wanneer de tijd richting middernacht gaat, zal het bioritme meer nood hebben aan slaap. Soms kan er geen consistentie aan het bioritme gehouden worden, bijvoorbeeld wanneer iemand nachtdienst heeft.
\subsection{Monotonie van de taak}
Vermoeidheid kan zich ook vormen doordat men een monotone taak uitvoert. Een taak is monotoon wanneer prikkels ontbreken, verandering erg voorspelbaar is of er een hoge maat van herhaling is \autocite{DaCoTA}. Het rijden op een autosnelweg met weinig verkeer en verandering van omgeving kan men zien als een monotone taak. Uit een experimenteel onderzoek is ook gebleken dat de monotonie van een rijtaak na verloop van tijd tot een slechtere rijprestatie leidt.
\subsection{Individuele kenmerken}
Tenslotte hebben individuele kenmerken ook een invloed op de vermoeidheid. Deze factoren, zoals leeftijd, medische conditie, het gebruik van alcohol, geneesmiddelen of drugs, beïnvloeden hoe vatbaar iemand is voor vermoeidheid en hoe goed ze er mee kunnen omgaan \autocite{VanSchagen2003}. Zo zijn oudere mensen, die last hebben van een slechte lichamelijke conditie, vaak sneller vermoeid. In tegenstelling tot tieners, die extra vatbaar zijn voor vermoeidheidseffecten door alcohol, drugs en te weinig slaap.

\section{Verband met verkeersveiligheid}
\subsection{Invloed op het rijgedrag}
Het effect van vermoeidheid op verkeersveiligheid is in verschillende studies onderzocht. Volgens deze studies leidt vermoeidheid achter het stuur tot een aantal negatieve effecten op verkeersgedrag \autocite{RiguelleGoldenbeld}. Deze bestaan uit een tragere reactietijd, verminderende oplettendheid en verwerking van informatie en slechter sturen \autocite{Bartlett,Friswell2008}. 

Volgens onderzoek is de rijprestatie van bestuurders na 17-19 uur slaapdeprivatie slechter dan die van bestuurders met een BAC (bloed-alcohol percentage) van 0.5\%, wat de wettelijke grens is in de meeste Europese landen en Australië \autocite{Williamson2000}. Onderzoek van Dawson \& Reid (1997) geeft aan dat de rijprestatie afneemt na 16 uur slaapdeprivatie en dat 21 uur leidt tot een verminderde rijprestatie die gelijk is aan een BAC van 0.8\%, wat het wettelijk limiet is in Engeland, de Verenigde Staten en Canada \autocite{Dawson1997}.
\subsection{Gevolgen in termen van ongevallen}
De prevalentie van vermoeidheid en de impact dat het heeft op ongevallen, is niet makkelijk te meten \autocite{Diependaele2015}. Dit komt doordat er geen betrouwbaar meetprotocol bestaat. Mensen die in een ongeval terecht komen, geven vaak ook niet toe dat het door de vermoeidheid komt dat het heeft plaatsgevonden. De gegevens uit het "100 Car Naturalistic Driving"-onderzoek tonen aan dat rijden als men moe is, leidt tot een vier keer meer risico op een ongeval \autocite{Klauer2006}.

Wetenschappelijke schattingen, gebaseerd op diepgaande analyses van de verkeersongevallen, geven aan dat 10 tot 15\% van de ongevallen te maken zou hebben met vermoeidheid \autocite{VanSchagen2003}. Een andere schatting zegt dat ongeveer 20 tot 25\% van de ongevallen op Europese wegen te maken heeft met vermoeidheid \autocite{Akerstedt2013}. \textcite{Horne1995} schatten dat vermoeidheid de oorzaak is van 16\% voor de ongevallen op stadswegen en 20\% op autosnelwegen. Een recentere studie van dezelfde auteurs \autocite{Horne1999} bevestigt dat de ongevallen door middel van vermoeidheid vaker gebeuren op autosnelwegen doordat rijden op een autosnelweg gezien wordt als een monotone taak. 
Gebleken uit een analyse van over 600 ongevallen met vrachtwagens in Europa blijkt dat vermoeidheid de hoofdoorzaak is van 6\% van de geanalyseerde ongevallen, waarvan maar liefst 37\% dodelijke ongevallen zijn \autocite{IRU2007}.

In België heeft een analyse van 125 letselongevallen met bussen of vrachtwagens aangetoond dat 10\% van de ongevallen veroorzaakt zijn door vermoeidheid \autocite{Herdewyn2010}. De Belgische ongevallenstatistieken van 2013 tonen aan dat het om 15.1\% van de verkeersongevallen om eenzijdige ongevallen, tegen een obstakel dat buiten de rijweg lag, ging. Hieruit kan er vermoed worden dat het om vermoeidheid ging.

Hoewel deze cijfers niet exact zijn, kan er wel geconcludeerd dat vermoeidheid toch wel een problematiek is binnenin de verkeersveiligheid. Zowel het ongevalsrisico en de ernst van het ongeval stijgen, zeker wanneer vermoeid rijden gecombineerd wordt met andere risicofactoren zoals alcohol, medicijnen en medische aandoeningen.

\section{Vermoeidheid detecteren}
Om een vermoeidheidsindicator te creëren, is er nood aan het effectief detecteren van vermoeidheid. Hiervoor is er al een algoritme die uit vijf fases bestaat. Deze zijn:
\begin{itemize}
    \item \textbf{Beeld voorbewerking}
    \item \textbf{Gezichtsdetectie}
    \item \textbf{Detecteren van het oog}
    \item \textbf{Knipperdetectie}
    \item \textbf{Oordelen van de vermoeidheid}
\end{itemize}
\autocite{Jibo2013}

\subsection{Beeld voorbewerking}
In de eerste fase wordt het beeld, dat de smartphone vastlegt, omgezet naar een kleinere resolutie. Vervolgens wordt het beeld getransformeerd naar het grijs. Doordat het beeld zijn resolutie gereduceerd wordt en het daarna naar het grijs wordt omgezet, wordt de data dat verwerkt moet worden gereduceerd. Dit zorgt voor een snellere werking van het detecteren.
\subsection{Haar-achtige kenmerkdetectie}
Om het gezicht en de ogen te kunnen waarnemen, wordt er gebruik gemaakt van Haar-achtige kenmerkdetectie. Het meest gebruikte is AdaBoost vanwege zijn snelheid en nauwkeurigheid \autocite{Viola2004}.

Deze kenmerkdetectie houdt rekening met aangrenzende rechthoeken binnen een specifiek gebied in een bewegend detectievenster. Een afbeeldingsgebied kan omschreven worden als een combinatie van verschillende Haar-achtige kenmerken. Het aantal en types kunnen dan weer aangezien worden als verschillende objecten. De cumulatieve som van intensiteit vanaf de oorsprong wordt gedefinieerd als \begin{equation*}S_{i,j} = \sum_{x=0}^{i} \sum_{y=0}^{j} I(i, j)\end{equation*} Hierbij staat \( I_\text{(i,j)} \) voor de intensiteit op locatie (i,j), en \( S_\text{(i,j)} \) voor de cumulatieve som van intensiteiten vanaf de oorsprong op locatie (i,j). De som van intensiteit van de rechthoek, gedefinieerd als twee punten \( (x_{\text{links}}, y_{\text{boven}}) \) en  \( (x_{\text{rechts}}, y_{\text{onder}}) \), kan berekend worden volgens deze formule, die de berekening versnelt \begin{equation*}
    S_{\text{acc}}(x_{\text{rechts}}, y_{\text{onder}}) - S_{\text{acc}}(x_{\text{links}}, y_{\text{onder}}) - S_{\text{acc}}(x_{\text{rechts}}, y_{\text{boven}}) + S_{\text{acc}}(x_{\text{links}}, y_{\text{boven}})
\end{equation*}
%%=============================================================================
%% Methodologie
%%=============================================================================

\chapter{\IfLanguageName{dutch}{Methodologie}{Methodology}}%
\label{ch:methodologie}

%% TODO: In dit hoofstuk geef je een korte toelichting over hoe je te werk bent
%% gegaan. Verdeel je onderzoek in grote fasen, en licht in elke fase toe wat
%% de doelstelling was, welke deliverables daar uit gekomen zijn, en welke
%% onderzoeksmethoden je daarbij toegepast hebt. Verantwoord waarom je
%% op deze manier te werk gegaan bent.
%% 
%% Voorbeelden van zulke fasen zijn: literatuurstudie, opstellen van een
%% requirements-analyse, opstellen long-list (bij vergelijkende studie),
%% selectie van geschikte tools (bij vergelijkende studie, "short-list"),
%% opzetten testopstelling/PoC, uitvoeren testen en verzamelen
%% van resultaten, analyse van resultaten, ...
%%
%% !!!!! LET OP !!!!!
%%
%% Het is uitdrukkelijk NIET de bedoeling dat je het grootste deel van de corpus
%% van je bachelorproef in dit hoofstuk verwerkt! Dit hoofdstuk is eerder een
%% kort overzicht van je plan van aanpak.
%%
%% Maak voor elke fase (behalve het literatuuronderzoek) een NIEUW HOOFDSTUK aan
%% en geef het een gepaste titel.
\section{Literatuurstudie}
In deze fase is er systematisch gezocht naar allerlei relevante literatuur met betrekking op het onderzoeksonderwerp. De literatuurstudie bevat informatie en technologieën die in verband staan met het onderzoek. Dit dient als een basis om er voor te zorgen dat de lezer een houvast krijgt van wat hij moet verwachten in dit onderzoek.

\section{Proof of Concept}
Nadat er genoeg informatie verkregen is, zal het ontwerp van de PoC tot stand komen. De PoC dient als een soort van bewijs om aan te tonen dat de oplossing van de onderzoeksvraag behaald kan worden. Er wordt eerst een ontwerp ontwikkelt en vervolgens wordt dit geïmplementeerd om aan te tonen dat het haalbaar is. Nadien kan de PoC geëvalueerd worden.

\section{Testcases}
Vervolgens volgen de testcases. Deze dienen om de PoC te testen en, indien nodig, wat bij te werken. De bedoeling van deze testcases is om er voor te zorgen dat de PoC zo nauwkeurig mogelijk en correct werkt. Dit wordt behaald door verschillende keren te kijken of de vermoeidheid wel effectief gedetecteerd wordt, inclusief verschillende omstandigheden die eventueel over het hoofd gezien zijn.

\section{Conclusie}
Na de uitvoering van de testcases en het bijwerken van de PoC volgt de conclusie. Dit baseert zich op wat de testcases opleveren en zo kan er een verdict komen om te kijken hoe deze technologie geïmplementeerd kan worden in het dagdagelijkse leven om het probleem van het onderzoeksvoorstel op te lossen.







% Voeg hier je eigen hoofdstukken toe die de ``corpus'' van je bachelorproef
% vormen. De structuur en titels hangen af van je eigen onderzoek. Je kan bv.
% elke fase in je onderzoek in een apart hoofdstuk bespreken.

%\input{...}
%\input{...}
%...

\input{conclusie}

%---------- Bijlagen -----------------------------------------------------------

\appendix

\chapter{Onderzoeksvoorstel}

Het onderwerp van deze bachelorproef is gebaseerd op een onderzoeksvoorstel dat vooraf werd beoordeeld door de promotor. Dat voorstel is opgenomen in deze bijlage.

%% TODO: 
%\section*{Samenvatting}

% Kopieer en plak hier de samenvatting (abstract) van je onderzoeksvoorstel.

% Verwijzing naar het bestand met de inhoud van het onderzoeksvoorstel
\input{../voorstel/voorstel-inhoud}

%%---------- Andere bijlagen --------------------------------------------------
% TODO: Voeg hier eventuele andere bijlagen toe. Bv. als je deze BP voor de
% tweede keer indient, een overzicht van de verbeteringen t.o.v. het origineel.
%\input{...}

%%---------- Backmatter, referentielijst ---------------------------------------

\backmatter{}

\setlength\bibitemsep{2pt} %% Add Some space between the bibliograpy entries
\printbibliography[heading=bibintoc]

\end{document}
