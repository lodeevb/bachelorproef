%%=============================================================================
%% Methodologie
%%=============================================================================

\chapter{\IfLanguageName{dutch}{Methodologie}{Methodology}}%
\label{ch:methodologie}

%% TODO: In dit hoofstuk geef je een korte toelichting over hoe je te werk bent
%% gegaan. Verdeel je onderzoek in grote fasen, en licht in elke fase toe wat
%% de doelstelling was, welke deliverables daar uit gekomen zijn, en welke
%% onderzoeksmethoden je daarbij toegepast hebt. Verantwoord waarom je
%% op deze manier te werk gegaan bent.
%% 
%% Voorbeelden van zulke fasen zijn: literatuurstudie, opstellen van een
%% requirements-analyse, opstellen long-list (bij vergelijkende studie),
%% selectie van geschikte tools (bij vergelijkende studie, "short-list"),
%% opzetten testopstelling/PoC, uitvoeren testen en verzamelen
%% van resultaten, analyse van resultaten, ...
%%
%% !!!!! LET OP !!!!!
%%
%% Het is uitdrukkelijk NIET de bedoeling dat je het grootste deel van de corpus
%% van je bachelorproef in dit hoofstuk verwerkt! Dit hoofdstuk is eerder een
%% kort overzicht van je plan van aanpak.
%%
%% Maak voor elke fase (behalve het literatuuronderzoek) een NIEUW HOOFDSTUK aan
%% en geef het een gepaste titel.
\section{Literatuurstudie}
In deze fase is er systematisch gezocht naar allerlei relevante literatuur met betrekking op het onderzoeksonderwerp. De literatuurstudie bevat informatie en technologieën die in verband staan met het onderzoek. Dit dient als een basis om er voor te zorgen dat de lezer een houvast krijgt van wat hij moet verwachten in dit onderzoek.

\section{Proof of Concept}
Nadat er genoeg informatie verkregen is, zal het ontwerp van de PoC tot stand komen. De PoC dient als een soort van bewijs om aan te tonen dat de oplossing van de onderzoeksvraag behaald kan worden. Er wordt eerst een ontwerp ontwikkelt en vervolgens wordt dit geïmplementeerd om aan te tonen dat het haalbaar is. Nadien kan de PoC geëvalueerd worden.

\section{Testcases}
Vervolgens volgen de testcases. Deze dienen om de PoC te testen en, indien nodig, wat bij te werken. De bedoeling van deze testcases is om er voor te zorgen dat de PoC zo nauwkeurig mogelijk en correct werkt. Dit wordt behaald door verschillende keren te kijken of de vermoeidheid wel effectief gedetecteerd wordt, inclusief verschillende omstandigheden die eventueel over het hoofd gezien zijn.

\section{Conclusie}
Na de uitvoering van de testcases en het bijwerken van de PoC volgt de conclusie. Dit baseert zich op wat de testcases opleveren en zo kan er een verdict komen om te kijken hoe deze technologie geïmplementeerd kan worden in het dagdagelijkse leven om het probleem van het onderzoeksvoorstel op te lossen.





