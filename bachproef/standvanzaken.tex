\chapter{\IfLanguageName{dutch}{Stand van zaken}{State of the art}}%
\label{ch:stand-van-zaken}

% Tip: Begin elk hoofdstuk met een paragraaf inleiding die beschrijft hoe
% dit hoofdstuk past binnen het geheel van de bachelorproef. Geef in het
% bijzonder aan wat de link is met het vorige en volgende hoofdstuk.

% Pas na deze inleidende paragraaf komt de eerste sectiehoofding.
Om dit onderzoek tot stand te brengen, is er nood aan bepaalde informatie, waaronder over AI-technieken en vermoeidheid zelf. Nadien wordt dit alles verwerkt om het probleem aan te pakken.

\section{Vermoeidheid}
De verkeersveiligheid heeft een grote last van vermoeidheid en slaperigheid. In de literatuur is de definitie van vermoeidheid en slaperigheid anders en bestaat er eigenlijk geen duidelijk onderscheid tussen beide \autocite{RiguelleGoldenbeld}.

Vermoeidheid verwijst naar een weerstand die veroorzaakt wordt door uitputting. Dit heeft het gevolg dat de taken minder efficiënt worden uitgevoerd. Dit kan zowel fysiek (bijvoorbeeld na het sporten of intensief werken) als mentaal zijn (na een veeleisende intellectuele, mentale of psychologische activiteit). Vermoeidheid zorgt er voor dat de taak minder snel wordt afgemaakt of dat de energie zodanig laag ligt dat er aan de volgende taak niet begonnen wordt.

Slaperigheid verwijst naar de moeite om nog wakker te blijven. Het is gekoppeld aan de biologische slaap-waak proces volgens het circadiaan-ritme. Het circadiaan-ritme is een endogene, regelbare schommeling van ongeveer 24 uur (van Latijnse `circa-diem` - ongeveer een dag) van verschillende biologische systemen in het hele lichaam \autocite{Gianni2018}. Slaperigheid heeft dus geen rechtstreeks verband met de uitvoering van een activiteit. Doordat ons menselijk lichaam beschikt over een 'slaapmodus', dit is voornamelijk tussen middernacht en 6 uur, neemt de alertheid bijgevolg af. Dit komt doordat het lichaam, op een cyclus van 24 uur, meer slaap nodig heeft dan op andere momenten.

Hoewel slaperigheid en vermoeidheid logisch niet synchroon verlopen, worden ze vaak gezamenlijk behandeld in de literatuur vanwege hun overeenkomstige gevolgen. Deze twee aparte gevallen kunnen ook tegelijkertijd voorkomen bij iemand. Om het eenvoudig te houden, wordt 'vermoeidheid' waargenomen als slaperigheid.

\section{Oorzaken van vermoeidheid}
Er zijn vijf algemene factoren, namelijk: de tijd besteed aan een taak of werk, slaaptekort, bioritme, monotonie van een taak en individuele kenmerken, die vermoeidheid veroorzaken \autocite{Brown}.
\subsection{De tijd besteed aan een taak of werk}
Een van de mogelijke oorzaken is de tijd die iemand besteed aan een bepaalde taak of werk. De meeste mensen voelen de eerste symptomen van fysieke vermoeidheid na ongeveer 2 à 3 uren aaneengesloten autorijden \autocite{RiguelleGoldenbeld}.
\subsection{Slaaptekort}
Een slaaptekort kan chronisch of acuut zijn \autocite{VanSchagen2003}. Een chronisch slaaptekort kan het veroorzaakt worden door gevolg van te weinig slaap over een lange periode. Men heeft gemiddeld nood aan 8 uur slaap. Echter is het ook van belang dat er goed geslapen wordt. Er kunnen regelmatige verstoringen optreden tijdens het slapen, wat ook leidt tot een chronisch slaaptekort. Een acuut slaaptekort komt ook door het te weinig slapen, maar is minder structureel. Na één slechte of korte nacht, wordt er al gesproken van een partieel acuut slaaptekort. Een volledig acuut slaaptekort komt pas voor indien er in de gehele 24-uursperiode niet geslapen is.
\subsection{Bioritme}
Het bioritme regelt het slaap-waakritme van de mens. Dit is bij iedereen anders. Het hangt dus samen met de dagelijkse slaapcyclus. Dit betekent dat de mens in de ochtend minder behoefte heeft aan slaapt dan op andere tijdstippen. Wanneer de tijd richting middernacht gaat, zal het bioritme meer nood hebben aan slaap. Soms kan er geen consistentie aan het bioritme gehouden worden, bijvoorbeeld wanneer iemand nachtdienst heeft.
\subsection{Monotonie van de taak}
Vermoeidheid kan zich ook vormen doordat men een monotone taak uitvoert. Een taak is monotoon wanneer prikkels ontbreken, verandering erg voorspelbaar is of er een hoge maat van herhaling is \autocite{DaCoTA}. Het rijden op een autosnelweg met weinig verkeer en verandering van omgeving kan men zien als een monotone taak. Uit een experimenteel onderzoek is ook gebleken dat de monotonie van een rijtaak na verloop van tijd tot een slechtere rijprestatie leidt.
\subsection{Individuele kenmerken}
Tenslotte hebben individuele kenmerken ook een invloed op de vermoeidheid. Deze factoren, zoals leeftijd, medische conditie, het gebruik van alcohol, geneesmiddelen of drugs, beïnvloeden hoe vatbaar iemand is voor vermoeidheid en hoe goed ze er mee kunnen omgaan \autocite{VanSchagen2003}. Zo zijn oudere mensen, die last hebben van een slechte lichamelijke conditie, vaak sneller vermoeid. In tegenstelling tot tieners, die extra vatbaar zijn voor vermoeidheidseffecten door alcohol, drugs en te weinig slaap.

\section{Verband met verkeersveiligheid}
\subsection{Invloed op het rijgedrag}
Het effect van vermoeidheid op verkeersveiligheid is in verschillende studies onderzocht. Volgens deze studies leidt vermoeidheid achter het stuur tot een aantal negatieve effecten op verkeersgedrag \autocite{RiguelleGoldenbeld}. Deze bestaan uit een tragere reactietijd, verminderende oplettendheid en verwerking van informatie en slechter sturen \autocite{Bartlett,Friswell2008}. 

Volgens onderzoek is de rijprestatie van bestuurders na 17-19 uur slaapdeprivatie slechter dan die van bestuurders met een BAC (bloed-alcohol percentage) van 0.5\%, wat de wettelijke grens is in de meeste Europese landen en Australië \autocite{Williamson2000}. Onderzoek van Dawson \& Reid (1997) geeft aan dat de rijprestatie afneemt na 16 uur slaapdeprivatie en dat 21 uur leidt tot een verminderde rijprestatie die gelijk is aan een BAC van 0.8\%, wat het wettelijk limiet is in Engeland, de Verenigde Staten en Canada \autocite{Dawson1997}.
\subsection{Gevolgen in termen van ongevallen}
De prevalentie van vermoeidheid en de impact dat het heeft op ongevallen, is niet makkelijk te meten \autocite{Diependaele2015}. Dit komt doordat er geen betrouwbaar meetprotocol bestaat. Mensen die in een ongeval terecht komen, geven vaak ook niet toe dat het door de vermoeidheid komt dat het heeft plaatsgevonden. De gegevens uit het "100 Car Naturalistic Driving"-onderzoek tonen aan dat rijden als men moe is, leidt tot een vier keer meer risico op een ongeval \autocite{Klauer2006}.

Wetenschappelijke schattingen, gebaseerd op diepgaande analyses van de verkeersongevallen, geven aan dat 10 tot 15\% van de ongevallen te maken zou hebben met vermoeidheid \autocite{VanSchagen2003}. Een andere schatting zegt dat ongeveer 20 tot 25\% van de ongevallen op Europese wegen te maken heeft met vermoeidheid \autocite{Akerstedt2013}. \textcite{Horne1995} schatten dat vermoeidheid de oorzaak is van 16\% voor de ongevallen op stadswegen en 20\% op autosnelwegen. Een recentere studie van dezelfde auteurs \autocite{Horne1999} bevestigt dat de ongevallen door middel van vermoeidheid vaker gebeuren op autosnelwegen doordat rijden op een autosnelweg gezien wordt als een monotone taak. 
Gebleken uit een analyse van over 600 ongevallen met vrachtwagens in Europa blijkt dat vermoeidheid de hoofdoorzaak is van 6\% van de geanalyseerde ongevallen, waarvan maar liefst 37\% dodelijke ongevallen zijn \autocite{IRU2007}.

In België heeft een analyse van 125 letselongevallen met bussen of vrachtwagens aangetoond dat 10\% van de ongevallen veroorzaakt zijn door vermoeidheid \autocite{Herdewyn2010}. De Belgische ongevallenstatistieken van 2013 tonen aan dat het om 15.1\% van de verkeersongevallen om eenzijdige ongevallen, tegen een obstakel dat buiten de rijweg lag, ging. Hieruit kan er vermoed worden dat het om vermoeidheid ging.

Hoewel deze cijfers niet exact zijn, kan er wel geconcludeerd dat vermoeidheid toch wel een problematiek is binnenin de verkeersveiligheid. Zowel het ongevalsrisico en de ernst van het ongeval stijgen, zeker wanneer vermoeid rijden gecombineerd wordt met andere risicofactoren zoals alcohol, medicijnen en medische aandoeningen.

\section{Vermoeidheid detecteren}
Om een vermoeidheidsindicator te creëren, is er nood aan het effectief detecteren van vermoeidheid. Hiervoor is er al een algoritme die uit vijf fases bestaat. Deze zijn:
\begin{itemize}
    \item \textbf{Beeld voorbewerking}
    \item \textbf{Gezichtsdetectie}
    \item \textbf{Detecteren van het oog}
    \item \textbf{Knipperdetectie}
    \item \textbf{Oordelen van de vermoeidheid}
\end{itemize}
\autocite{Jibo2013}

\subsection{Beeld voorbewerking}
In de eerste fase wordt het beeld, dat de smartphone vastlegt, omgezet naar een kleinere resolutie. Vervolgens wordt het beeld getransformeerd naar het grijs. Doordat het beeld zijn resolutie gereduceerd wordt en het daarna naar het grijs wordt omgezet, wordt de data dat verwerkt moet worden gereduceerd. Dit zorgt voor een snellere werking van het detecteren.
\subsection{Haar-achtige kenmerkdetectie}
Om het gezicht en de ogen te kunnen waarnemen, wordt er gebruik gemaakt van Haar-achtige kenmerkdetectie. Het meest gebruikte is AdaBoost vanwege zijn snelheid en nauwkeurigheid \autocite{Viola2004}.

Deze kenmerkdetectie houdt rekening met aangrenzende rechthoeken binnen een specifiek gebied in een bewegend detectievenster. Een afbeeldingsgebied kan omschreven worden als een combinatie van verschillende Haar-achtige kenmerken. Het aantal en types kunnen dan weer aangezien worden als verschillende objecten. De cumulatieve som van intensiteit vanaf de oorsprong wordt gedefinieerd als \begin{equation*}S_{i,j} = \sum_{x=0}^{i} \sum_{y=0}^{j} I(i, j)\end{equation*} Hierbij staat \( I_\text{(i,j)} \) voor de intensiteit op locatie (i,j), en \( S_\text{(i,j)} \) voor de cumulatieve som van intensiteiten vanaf de oorsprong op locatie (i,j). De som van intensiteit van de rechthoek, gedefinieerd als twee punten \( (x_{\text{links}}, y_{\text{boven}}) \) en  \( (x_{\text{rechts}}, y_{\text{onder}}) \), kan berekend worden volgens deze formule, die de berekening versnelt \begin{equation*}
    S_{\text{acc}}(x_{\text{rechts}}, y_{\text{onder}}) - S_{\text{acc}}(x_{\text{links}}, y_{\text{onder}}) - S_{\text{acc}}(x_{\text{rechts}}, y_{\text{boven}}) + S_{\text{acc}}(x_{\text{links}}, y_{\text{boven}})
\end{equation*}