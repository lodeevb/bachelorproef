\chapter{\IfLanguageName{dutch}{Stand van zaken}{State of the art}}%
\label{ch:stand-van-zaken}

% Tip: Begin elk hoofdstuk met een paragraaf inleiding die beschrijft hoe
% dit hoofdstuk past binnen het geheel van de bachelorproef. Geef in het
% bijzonder aan wat de link is met het vorige en volgende hoofdstuk.

% Pas na deze inleidende paragraaf komt de eerste sectiehoofding.
Om dit onderzoek tot stand te brengen, is er nood aan bepaalde informatie. Dit bevat informatie omtrent de huidige verkeersveiligheid en AI-technieken. Zo kan er informatie vergaart worden over hoe dat dit probleem aangepakt kan worden.

\section{Vermoeidheid}
De verkeersveiligheid heeft een grote last van vermoeidheid en slaperigheid. In de literatuur is de definitie van vermoeidheid en slaperigheid anders en bestaat er eigenlijk geen duidelijk onderscheid tussen beide \autocite{RiguelleGoldenbeld}.

Vermoeidheid verwijst naar een weerstand die veroorzaakt wordt door uitputting. Dit heeft het gevolg dat de taken minder efficiënt worden uitgevoerd. Dit kan zowel fysiek (bijvoorbeeld na het sporten of intensief werken) als mentaal zijn (na een veeleisende intellectuele, mentale of psychologische activiteit). Vermoeidheid zorgt er voor dat de taak minder snel wordt afgemaakt of dat de energie zodanig laag ligt dat er aan de volgende taak niet begonnen wordt.

Slaperigheid verwijst naar de moeite om nog wakker te blijven. Het is gekoppeld aan de biologische slaap-waak proces volgens het circadiaan-ritme. Het circadiaan-ritme is een endogene, regelbare schommeling van ongeveer 24 uur (van Latijnse `circa-diem` - ongeveer een dag) van verschillende biologische systemen in het hele lichaam \autocite{Gianni2018}. Slaperigheid heeft dus geen rechtstreeks verband met de uitvoering van een activiteit. Doordat ons menselijk lichaam beschikt over een 'slaapmodus', dit is voornamelijk tussen middernacht en 6 uur, neemt de alertheid bijgevolg af. Dit komt doordat het lichaam, op een cyclus van 24 uur, meer slaap nodig heeft dan op andere momenten.

Hoewel slaperigheid en vermoeidheid logisch niet synchroon verlopen, worden ze vaak gezamenlijk behandeld in de literatuur vanwege hun overeenkomstige gevolgen. Deze twee aparte gevallen kunnen ook tegelijkertijd voorkomen bij iemand. Om het eenvoudig te houden, wordt 'vermoeidheid' waargenomen als slaperigheid.
