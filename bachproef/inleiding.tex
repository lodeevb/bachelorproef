%%=============================================================================
%% Inleiding
%%=============================================================================

\chapter{\IfLanguageName{dutch}{Inleiding}{Introduction}}%
\label{ch:inleiding}

\section{\IfLanguageName{dutch}{Probleemstelling}{Problem Statement}}%
\label{sec:probleemstelling}

In onze moderne en snel voortbewegende wereld is vermoeidheid een groeiend probleem. Dit heeft een grote impact op de verkeersveiligheid. Het aantal ongevallen die in verband staan met vermoeidheid zijn alarmerend hoog. Om dit probleem op te lossen zullen er innovatieve oplossingen en technologieën nodig zijn die de vermoeidheid vroegtijdig kan opsporen om zo te voorkomen dat automobilisten het verkeer in gevaar brengen. 

\section{\IfLanguageName{dutch}{Onderzoeksvraag}{Research question}}%
\label{sec:onderzoeksvraag}

Dit onderzoek zal dus achterhalen in welke mate er kunstmatige intelligentie geïmplementeerd kan worden zodat de vermoeidheid tijdig wordt opgespoord. Vervolgens zal er verder onderzocht worden hoe dat deze mogelijke oplossingen ingezet kunnen worden in de toekomst om zo het aantal ongevallen die veroorzaakt worden door vermoeidheid te verminderen.

\section{\IfLanguageName{dutch}{Onderzoeksdoelstelling}{Research objective}}%
\label{sec:onderzoeksdoelstelling}

Het belangrijkste doel van dit onderzoek is het creëren en beoordelen van een Proof-of-Concept. Deze PoC zal gebruik maken van kunstmatige intelligentie die de vermoeidheid kan detecteren en signaleren. Het uiteindelijke doel is om mogelijke oplossingen te onderzoeken die eventueel deze technologie kan toepassen om zo verkeersongevallen te verminderen. Het onderzoek zal zich specifiek richten op:
\begin{itemize}
    \item \textbf{Proof-of-Concept:}
    Het ontwerpen en implementeren van een AI-systeem die de vermoeidheid kan detecteren door middel van relevante indicatoren zoals gezichtsuitdrukkingen, oogbewegingen, etc.
    \item \textbf{Evaluatie:}
    Het uitvoeren van tests en experimenten om de effectiviteit van de Proof-of-Concept te evalueren in verschillende
    scenario's.
    \item \textbf{Maatregelen:}
    Het onderzoeken naar mogelijke maatregelen op basis van de verzamelde gegevens en resultaten om zo ongevallen te voorkomen.
    \item \textbf{Ethiek en Privacy:}
    Het beschermen van de privacy van automobilisten door middel van ethische en privacy-gerelateerde overwegingen te onderzoeken.
\end{itemize} 

Het onderzoek dient als een basis voor verdere onderzoeken en implementaties op het gebied van de detectie van vermoeidheid tijdens het rijden. De bevindingen zullen bijdragen aan de toekomstige implementatie van kunstmatige intelligentie in het verkeer.

\section{\IfLanguageName{dutch}{Opzet van deze bachelorproef}{Structure of this bachelor thesis}}%
\label{sec:opzet-bachelorproef}

% Het is gebruikelijk aan het einde van de inleiding een overzicht te
% geven van de opbouw van de rest van de tekst. Deze sectie bevat al een aanzet
% die je kan aanvullen/aanpassen in functie van je eigen tekst.

De rest van deze bachelorproef is als volgt opgebouwd:

In Hoofdstuk~\ref{ch:stand-van-zaken} wordt een overzicht gegeven van de stand van zaken binnen het onderzoeksdomein, op basis van een literatuurstudie.

In Hoofdstuk~\ref{ch:methodologie} wordt de methodologie toegelicht en worden de gebruikte onderzoekstechnieken besproken om een antwoord te kunnen formuleren op de onderzoeksvragen.

In Hoofdstuk~\ref{ch:proof-of-concept} wordt de Proof-of-Concept ontwikkelt die gebruik maakt van kunstmatige intelligentie om vermoeidheid te detecteren.

In Hoofdstuk~\ref{ch:testcases} worden er verschillende testcases uitgevoerd die bepaalde automobilisten gaan controleren op vermoeidheid in het verkeer.

In Hoofdstuk~\ref{ch:conclusie}, tenslotte, wordt de conclusie gegeven en een antwoord geformuleerd op de onderzoeksvragen. Daarbij wordt ook een aanzet gegeven voor toekomstig onderzoek binnen dit domein.