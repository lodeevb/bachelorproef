%%=============================================================================
%% Voorwoord
%%=============================================================================

\chapter*{\IfLanguageName{dutch}{Woord vooraf}{Preface}}%
\label{ch:voorwoord}
\setlength{\parskip}{1em}
%% TODO:
%% Het voorwoord is het enige deel van de bachelorproef waar je vanuit je
%% eigen standpunt (``ik-vorm'') mag schrijven. Je kan hier bv. motiveren
%% waarom jij het onderwerp wil bespreken.
%% Vergeet ook niet te bedanken wie je geholpen/gesteund/... heeft
Geachte lezer,

Met veel genoegen presenteer ik mijn bachelorproef, met de titel "In hoeverre kan kunstmatige intelligentie worden in-gezet om vermoeidheid bij automobilisten te detecteren en signaleren, en welke maatregelen zouden kunnen worden geïmplementeerd om ongevallen te voor-komen?".
Dit onderzoek is het resultaat van mijn toewijding aan het vakgebied van kunstmatige intelligentie.

De keuze voor dit onderwerp werd beslist door mijn interesse in de opkomst van kunstmatige intelligentie en de bezorgdheid over het gevaarlijk rijgedrag van sommige automobilisten. In dit voorwoord
wil ik kort de achtergrond van mijn onderzoek uitleggen en mijn dank uitspreken aan degene die mij hebben geholpen aan het tot stand komen van dit werk.

Mijn oprechte dank gaat uit naar mijn begeleider, Alec Hantson, voor de waardevolle begeleiding en ondersteuning gedurende dit onderzoeksproces. Zijn expertise op het vakgebied van kunstmatige intelligentie heeft mij veel begrip en inzicht gegeven.
Ook wil ik mijn promotor, Chloé De Leenheer, bedanken die mij ondersteunt en begeleidt heeft om dit werk zo correct en professioneel mogelijk te verwoorden.

Dit onderzoek heeft niet enkel mijn academische vaardigheid verbreedt, maar heeft mij ook bewust gemaakt op de impact die kunstmatige intelligentie kan hebben op maatschappelijke kwesties zoals verkeersveiligheid.
Ik hoop dat de resultaten en aanbevelingen in deze bachelorproef bijdragen aan de rol van technologie in het verkeerd.

Ik nodig u uit om de pagina's van dit werk te verkennen en hoop dat dit u een inzicht geeft in de relatie tussen kunstmatige intelligentie, vermoeidheid bij automobilisten en verkeersveiligheid.
Hopelijk draagt dit onderzoek bij aan een veiligere en slimmere mobiliteit in het verkeer in de toekomst.

Met vriendelijke groet,

Lode Van Beneden
\\
Bachelor Toegepaste informatica
\\
Hogeschool Gent